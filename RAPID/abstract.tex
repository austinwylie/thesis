Several hundred thousand miles of energy pipelines span the whole of North America---responsible for carrying the natural gas and liquid petroleum that power the continent's homes and economies~\cite{PHMSA}. These pipelines, so crucial to everyday goings-on, are closely monitored by various operating companies to ensure they perform safely and smoothly~\cite{PHMSA2013}. 

Happenings like earthquakes, erosion, and extreme weather, however---and human factors like vehicle traffic and construction---all pose threats to pipeline integrity. As such, there is a tremendous need to measure and indicate useful, actionable data for each region of interest, and operators often use computer-based decision support systems (DSS) to analyze and allocate resources for active and potential hazards~\cite{PHMSA2013,MichaelBakerJr.2008,Chastain,Dunning2013}.

We designed and implemented a geospatial data service, \textit{REST API for Pipeline Integrity Data} (RAPID) to improve the amount and quality of data available to DSS. More specifically, RAPID---built with a spatial database and the Django web framework---allows third-party software to manage and query an arbitrary number of geographic data sources through one centralized REST API.

Here, we focus on the process and peculiarities of creating RAPID's model and query interface for pipeline integrity management; this contribution describes the design, implementation, and validation of that model, which builds on existing geographic standards.